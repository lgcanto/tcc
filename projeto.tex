\documentclass[grad,numbers]{coppe}
\usepackage{amsmath,amssymb}
\usepackage{hyperref}
\usepackage[utf8]{inputenc}
\usepackage[brazil]{babel}
\usepackage[T1]{fontenc}
\usepackage{graphicx}

\makelosymbols
\makeloabbreviations

\begin{document}
  \title{Análise de Notícias do Mercado Financeiro Utilizando Processamento de Linguagem Natural e Aprendizado de Máquina Para Decisões de Swing Trade}
  \foreigntitle{Financial Market News Analysis Using Natural Language Processing and Machine Learning for Swing Trade Decisions}
  \author{Lucas}{Gama Canto}
  \advisor{Prof.}{Heraldo Luís}{Silveira de Almeida}{D.Sc.}

  \examiner{Prof.}{[TODO]Nome do Primeiro Examinador Sobrenome}{D.Sc.}
  \examiner{Prof.}{[TODO]Nome do Segundo Examinador Sobrenome}{Ph.D.}
  \examiner{Prof.}{[TODO]Nome do Terceiro Examinador Sobrenome}{D.Sc.}
  
  \department{ECA}
  
  \date{03}{2020}

  \keyword{Aprendizado de Máquina}
  \keyword{Processamento de Linguagem Natural}
  \keyword{Mercado Financeiro}

  \maketitle

  \frontmatter
  
  \makecatalog
  
  \dedication{Ao povo brasileiro, pela total contribuição em minha graduação.}

  \chapter*{Agradecimentos}

	  \paragraph{}Gostaria de agradecer a todas as pessoas e situações que tornaram este momento possível. Em especial, meus pais Benedita e Manoel, pelo suporte e esforço incondicional em apoiar minha decisão de vir estudar engenharia no Rio de Janeiro, aos professores da graduação, que me fizeram evoluir no âmbito acadêmico, profissional e pessoal, em especial ao meu orientador e professor Heraldo, que não mediu esforços para me ajudar neste trabalho, e aos amigos que me apoiaram e participaram do meu processo de graduação.

  \begin{abstract}

	  \paragraph{}Com o objetivo de automatizar análises fundamentalistas de mercado, o uso de tecnologia para processamento de texto vem sendo utilizado constantemente no meio acadêmico\cite{nlp-academics} e profissional\cite{nlp-industry}. De forma a contribuir para este campo em crescimento, este trabalho discorre um estudo acerca da criação de modelos preditivos sobre a valorização ou desvalorização de ações na bolsa de valores do Brasil (B3, antiga Bovespa) a partir de notícias sobre o mercado brasileiro de forma a auxiliar decisões de Swing Trade, ou seja, compra e venda de ações dentro de uma janela de tempo maior que um dia.
	  \paragraph{}Para isto, o presente projeto utiliza o framework PyText, que se baseia em conceitos de Aprendizado de Máquina, Redes Neurais e Processamento de Linguagem Natural de forma a desenvolver modelos preditivos com a tarefa de classificação textual.

  \end{abstract}

  \begin{foreignabstract}

	  \paragraph{}In order to automate fundamental market analysis, the use of text processing technology has been constantly used in academic\cite{nlp-academics} and professional\cite{nlp-industry} means. To contribute to this growing field, this paper discusses a study about the creation of predictive models regarding the valuation or devaluation of shares on the Brazilian stock exchange (B3, former Bovespa) based on news about the Brazilian market in order to assist Swing Trade decisions, that is, buying and selling stocks within a time window longer than one day.
	  \paragraph{}To this end, the present project uses the PyText framework, which is based on Machine Learning, Neural Networks and Natural Language Processing concepts in order to develop predictive models with the task of textual classification.

  \end{foreignabstract}

  \tableofcontents
  \listoffigures
  \listoftables
  \printlosymbols
  \printloabbreviations

  \mainmatter
%  \doublespacing
	\chapter{Introdução}
		
		\section{Tema}
			\paragraph{}O tema deste trabalho se resume no estudo da criação de modelos preditivos de modo que estes possam prever a valorização ou desvalorização de ações da bolsa de valores por meio do processamento de notícias do mercado brasileiro.
			\paragraph{}Deste modo, o problema a ser abordado é a identificação de quando uma notícia pode impactar positivamente ou negativamente a variação de preço de ações de forma automatizada.
			
		\section{Delimitação}
			\paragraph{}Este trabalho se restringe ao processamento de texto em português brasileiro, tendo como foco a predição da variação de preço das ações que fazem parte da bolsa de valores do Brasil, a B3. Pela indisponibilidade de dados sobre notícias brasileiras contendo a informação do horário de lançamento da notícia, o projeto mira em predições dentro de uma janela de tempo maior que um dia, de forma a auxiliar decisões de Swing Trade, isto é, operações de compra e venda de ações numa janela de tempo maior que um dia.
			\paragraph{}Além disso, o estudo se baseia na ferramenta PyText, um framework recentemente desenvolvido pelo Facebook que providencia modelos de processamento de linguagem natural de última geração através de uma interface simples e extensível\cite{pytext-paper}.
		
		\section{Justificativa}
			\paragraph{}Diante do crescente número de investidores na bolsa de valores no Brasil, nota-se uma maior preocupação da população brasileira acerca da busca por independência financeira e fontes alternativas de renda com o intuito de contribuir à economia familiar, previdência, ou mesmo utilizar este método como fonte principal de renda\cite{bovespa-investors-growth}.
			\paragraph{}Ao mesmo tempo, estudos associados à inteligência artificial, aprendizado de máquina e processamento de linguagem natural continuam emergindo no meio acadêmico e auxiliando o meio profissional como nunca antes, incluindo o mercado financeiro\cite{ai-in-financial-growth}.
			\paragraph{}Através destes dois fatores, o presente trabalho busca contribuir para a difusão do estudo e uso de algumas destas tecnologias sobre um assunto que gradualmente se encontra dentro do interesse da população brasileira e que colabora para uma possível instauração de uma cultura de economia e independência financeira no Brasil.
		
		\section{Objetivos}
			\paragraph{}O objetivo geral do presente trabalho é de analisar modelos preditivos associados ao mercado financeiro que possam ser construídos a partir do framework PyText, tendo como objetivos específicos, apresentar: (1) A busca por dados de notícias e do histórico da bolsa de valores; (2) A lógica utilizada para a união destes dados de forma a construir os conjuntos de dados utilizados no treinamento dos modelos; (3) O pré-processamento dos conjuntos de dados; (4) As possíveis configurações do framework utilizado de forma a obter a melhor performance; (5) O detalhamento e a análise dos modelos finais encontrados.
		
		\section{Metodologia}
			\paragraph{}O trabalho teve início a partir da procura por bases de dados de notícias associadas ao mercado brasileiro e escritas em português do Brasil, seguida pela obtenção do histórico das variações de preço dos ativos que compõem o iBovespa. Após isto, o histórico foi filtrado de forma a manter as informações dos 5 ativos mais significativos e das varições destes ativos que ocorreram dentro da mesma janela de tempo das notícias obtidas. Em seguida, estes dados foram unidos de forma a obter 5 conjuntos de dados para cada ativo, cada um levando em consideração uma diferente janela de tempo para indicar a valorização: de 1 a 5 dias.
			\paragraph{}Logo após, houve a etapa de pré-processamento do corpo das notícias de forma a remover possíveis ruídos e facilitar a etapa de treinamento, sem perda de contexto do conteúdo. Com os conjuntos de dados prontos, foram feitos testes no PyText com o objetivo de definir a melhor configuração possível para a natureza dos dados, e assim obter a melhor performance.
			\paragraph{}Por fim, os testes finais de cada modelo gerado foi detalhado e analisado para permitir uma conclusão e avaliação do processo como um todo.

		\section{Descrição}
			\paragraph{}O capítulo 2 apresenta toda a fundamentação teórica utilizada como base para o projeto a partir de uma breve descrição de como a bolsa de valores funciona e como pode-se obter lucro a partir da mesma, seguida de explicações sobre Aprendizado de Máquina, Processamento de Linguagem Natural, Redes Neurais e o framework Pytext.
			\paragraph{}No capítulo 3 é detalhado todo o processo executado para obtenção do conjunto de notícias e do histórico da B3, seguido do pré-processamento realizado nestes dois conjuntos e a criação dos conjuntos de dados finais utilizados para o treino, cada um associado a um ativo e uma janela de tempo específica.
			\paragraph{}Os detalhes das configurações utilizadas no PyText e o treinamento em si é especificado no capítulo 4, onde há uma discussão acerca dos parâmetros encontrados para a geração de modelos mais performáticos, além das métricas finais encontradas para cada modelo gerado.
			\paragraph{}Por fim, o capítulo 5 apresenta uma conclusão acerca dos modelos encontrados seguido por sugestões que futuramente podem ser aplicadas para a evolução do tema e uma possível melhora de desempenho dos modelos preditivos.
			\paragraph{}O código desenvolvido para o pré-processamento e geração dos conjuntos de dados e arquivos de configurações do PyText utilizados para a geração dos modelos podem ser encontrados no repositório do github referenciado em \cite{github}.
   
  \chapter{Fundamentação Teórica}
  
  \section{Bolsa de Valores e Ações}
  	\paragraph{}A Bolsa de Valores é um lugar centralizado onde, além de abranger outros tipos de investimento, se negociam ações (também chamados de ativos ou papéis), isto é, parcelas do capital social de empresas de capital aberto. Atualmente a B3 (Brasil, Bolsa, Balcão) é a Bolsa de Valores oficial do Brasil que em 2017 atingiu a 5ª posição das maiores bolsas de mercados de capitais do mundo em valor de mercado, com um patrimônio de US\$ 13 bilhões\cite{b3-patrimonio}.
  	\paragraph{}As ações são negociadas diariamente a partir das ordens de compra e venda emitidas pelas corretoras durante o pregão eletrônico, que na B3, funciona em dias úteis das 10:00 às 17:00.
  	\subsection{Preços de Ações}
  		\paragraph{}O preço de um ativo na Bolsa de Valores pode ser determinado por diversas razões que podem se relacionar entre si, entre essas, pode-se destacar a lei da oferta e demanda, perspectivas de crescimento da empresa associada ao papel e especulação. A previsibilidade acerca de movimentações no mercado de ações normalmente pode ser baseada em Análise Técnica (estudo dos movimentos do mercado baseado em métricas como preço, volume e taxa de juros\cite{analise-tecnica}), Análise Fundamentalista (estudo feito a partir de resultados financeiros e operacionais, indicando a saúde da empresa\cite{analise-fundamentalista})
  
  \section{Aprendizado de Máquina}
  \section{Processamento de Linguagem Natural}
  \section{Redes Neurais}
  \section{PyText}
  
  \chapter{Obtenção e Pré-processamento de Dados}
  
  
  
  \chapter{Treinamento}
  
  
  
  \chapter{Conclusões}
  
  Segundo a norma de formatação de teses e dissertações do
  Instituto Alberto Luiz Coimbra de Pós-graduação e Pesquisa de
  Engenharia (COPPE), toda abreviatura deve ser definida antes de
  utilizada.\abbrev{COPPE}{Instituto Alberto Luiz Coimbra de Pós-graduação e Pesquisa de Engenharia}
  
  Do mesmo modo, é imprescindível definir os símbolos, tal como o
  conjunto dos números reais $\mathbb{R}$ e o conjunto vazio $\emptyset$.
  \symbl{$\mathbb{R}$}{Conjunto dos números reais}
  \symbl{$\emptyset$}{Conjunto vazio}
  
  Você deve selecionar seu curso de engenharia usando o comando \texttt{\textbackslash department\{Sigla\}} e no lugar de Sigla inserir a sigla referente ao seu curso de engenharia. A tabela \ref{tab:courses} relaciona as siglas dos cursos de engenharia da Escola Politécnica da Universidade Federal do Rio de Janeiro (POLI-UFRJ), enquanto que a tabela \ref{tab:programs} relaciona as siglas dos programas de pós graduação da COPPE.\abbrev{POLI-UFRJ}{Escola Politécnica da Universidade Federal do Rio de Janeiro}
  
  
  \begin{table}[h]
  	\caption{Siglas dos cursos de engenharia da Escola Politécnica da UFRJ.}
  	\label{tab:courses}
  	\centering
  	{\footnotesize
  		\begin{tabular}{|c|c|}
  			\hline
  			Sigla & Curso\\
  			\hline
  			EA &  Engenharia Ambiental \\
  			ECV & Engenharia Civil\\
  			ECI & Engenharia de Computação e Informação \\
  			ECA & Engenharia de Controle e Automação \\
  			EMAT & Engenharia de Materiais\\
  			EPT & Engenharia de Petróleo\\
  			EPR & Engenharia de Produção\\
  			EEC & Engenharia Eletrônica e de Computação\\
  			EET & Engenharia Elétrica\\
  			EMC & Engenharia Mecânica\\
  			EMET & Engenharia Metalúrgica\\
  			ENO & Engenharia Naval e Oceânica\\
  			ENU & Engenharia Nuclear\\
  			\hline
  	\end{tabular}}
  \end{table}
  
  
  \begin{table}[h]
  	\caption{Siglas dos programas de pós graduação da COPPE.}
  	\label{tab:programs}
  	\centering
  	{\footnotesize
  		\begin{tabular}{|c|c|}
  			\hline
  			Sigla & Curso\\
  			\hline
  			PEB & Engenharia Biomédica \\
  			PEC & Engenharia Civil\\
  			PEE & Engenharia Elétrica \\
  			PEM & Engenharia Mecânica \\
  			PEMM & Engenharia Metalúrgica e de Materiais\\
  			PEN & Engenharia Nuclear\\
  			PENO & Engenharia Oceânica\\
  			PPE & Planejamento Energético\\
  			PEP & Engenharia de Produção\\
  			PEQ & Engenharia Química\\
  			PESC & Engenharia de Sistemas e Computação\\
  			PET & Engenharia de Transportes\\
  			\hline
  	\end{tabular}}
  \end{table}
  
  
  Note também que todas as figuras ou tabelas devem ser citadas no texto. Como ocorre com as tabelas \ref{tab:courses} e \ref{tab:programs}. Para ilustrar o uso de figuras em \LaTeX, considere as figuras \ref{fig:poli} e \ref{fig:coppe}.
  
  \begin{figure}
  	\centering
  	\includegraphics[width=5cm]{poli-logo.pdf}
  	\caption{Logotipo da POLI-UFRJ.}
  	\label{fig:poli}
  \end{figure}
  
  \begin{figure}
  	\centering
  	\includegraphics[width=5cm]{coppe-logo.pdf}
  	\caption{Logotipo da COPPE.}
  	\label{fig:coppe}
  \end{figure}
  
  \chapter{Revisão Bibliográfica}

  Para ilustrar a completa adesão ao estilo de citações e listagem de
  referências bibliográficas, a Tabela \ref{tab:citation} apresenta citações de alguns dos trabalhos contidos na norma fornecida pela CPGP da
  COPPE, utilizando o estilo numérico.

  \begin{table}[h]
  \caption{Exemplos de citações utilizando o comando padrão
    \texttt{\textbackslash cite} do \LaTeX\ e
    o comando \texttt{\textbackslash citet},
    fornecido pelo pacote \texttt{natbib}.}
  \label{tab:citation}
  \centering
  {\footnotesize
  \begin{tabular}{|c|c|c|}
    \hline
    Tipo da Publicação & \verb|\cite| & \verb|\citet|\\
    \hline
    Livro & \cite{book-example} & \citet{book-example}\\
    Artigo & \cite{article-example} & \citet{article-example}\\
    Relatório & \cite{techreport-example} & \citet{techreport-example}\\
    Relatório & \cite{techreport-exampleIn} & \citet{techreport-exampleIn}\\
    Anais de Congresso & \cite{inproceedings-example} &
      \citet{inproceedings-example}\\
    Séries & \cite{incollection-example} & \citet{incollection-example}\\
    Em Livro & \cite{inbook-example} & \citet{inbook-example}\\
    Dissertação de mestrado & \cite{mastersthesis-example} &
      \citet{mastersthesis-example}\\
    Tese de doutorado & \cite{phdthesis-example} & \citet{phdthesis-example}\\
    \hline
  \end{tabular}}
  \end{table}
  
  É importante notar que, segundo a \href{http://www.poli.ufrj.br/graduacao_projeto.php}{Norma para a Elaboração Gráfica do Projeto de Graduação} da Escola Politécnica da UFRJ para trabalhos de conclusão de curso de engenharia de julho de 2012, as referências bibliográficas podem ser apresentadas de duas formas: $(i)$ Referências numeradas e $(ii)$ Referências em ordem alfabética. Para exibição numerada, em que a exibição das referências bibliográficas segue a ordem de citação usada no texto, use o comando \texttt{\textbackslash bibliographystyle\{coppe-unsrt\}}. Para exibição de referências bibliográficas em ordem alfabética, basta usar o comando \texttt{\textbackslash bibliographystyle\{coppe-plain\}} ao final do documento. 

  \backmatter
  \bibliographystyle{coppe-unsrt}
  \bibliography{bibliography}

  \appendix
  \chapter{Algumas Demonstrações}
\end{document}
%% 
%%
%% End of file `example.tex'.
